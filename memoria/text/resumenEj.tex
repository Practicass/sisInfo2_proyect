\section{Resumen Ejecutivo}
\paragraph{}
El presente documento presenta una evaluación detallada de la implementación de Odoo, un sistema de gestión empresarial de código abierto, en UZ-On-Marketing. Con el objetivo de optimizar la gestión de sus recursos y procesos, la empresa ha optado por esta solución modular, que ofrece una amplia gama de aplicaciones para abordar diversas áreas funcionales de la organización.

\paragraph{}
En este documento se realiza una evaluación del sistema de información en su totalidad, abordando las diversas funcionalidades que nos ofrece este ERP y que se adaptan a nuestras necesidades actuales y futuras. Se ha descubierto el gran potencial que nos brinda Odoo con su coste cero, lo cual, es muy complicado encontrar en el mercado actual. Vamos a comprobar si este factor afecta a sus funcionalidades por su reducido coste, o si por el contrario no merece la pena invertir tiempo y recursos en implementarlo en UZ-on-Marketing.

\paragraph{}
Los resultados obtenidos son muy diversos, en función de cada módulo. Nos hemos encontrado módulos muy sencillos de utilizar y con un gran potencial para llevar nuestra empresa al siguiente nivel, como el modulo \textit{Proyectos}. Por el contrario otros módulos con grandes complicaciones que nos impediría llegar a nuestros objetivos como es el módulo de contabilidad y finanzas (FICO). Somos conscientes que este tipo de software tiene sus limitaciones y es por eso que queremos averiguar cuales son y cómo afectarían a nuestra empresa.

\paragraph{}
Tras tres meses evaluando este sistema de información se ha llegado a la conclusión que Odoo es un ERP robusto, con un gran número de funcionalidades y puede potenciar nuestra compañía para lograr nuestros objetivos. Cabe destacar que en caso de adquirir este software se debe acompañar por una formación que subsane las limitaciones y deficiencias que presenta, como la pronunciada curva de aprendizaje y lo poco intuitiva que es la interfaz en ciertos módulos. 