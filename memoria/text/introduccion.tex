\section{Introducción}
\paragraph{}
Odoo se ha consolidado como un software de gestión empresarial de código abierto líder en el mercado. Su popularidad radica en su enfoque modular y su amplia gama de aplicaciones, que abarcan desde contabilidad y recursos humanos hasta ventas, inventario y CRM, entre otros. Esta versatilidad permite a las empresas adaptar el sistema a sus necesidades específicas, optimizando así sus procesos y centralizando su gestión de manera eficiente.
\paragraph{}
La naturaleza modular de Odoo no solo ofrece flexibilidad, sino también la capacidad de personalización que las empresas necesitan en un entorno empresarial en constante evolución. Con una interfaz intuitiva y accesible, Odoo facilita la adopción y el uso por parte de los usuarios finales, promoviendo la innovación continua dentro de las organizaciones.
\paragraph{}
En un mercado donde la agilidad y la capacidad de adaptación son clave para el éxito, Odoo se posiciona como una herramienta invaluable. Su enfoque de código abierto no solo reduce los costos de implementación, sino que también brinda a las empresas la libertad de explorar y desarrollar soluciones a medida según sus necesidades específicas.
\paragraph{}
En nuestro contexto, UZ-On-Marketing se enfrenta al desafío de seleccionar un sistema de planificación de recursos empresariales (ERP) que se adapte a sus necesidades y presupuesto. Considerando diversas opciones como SAP, SAGE y Odoo, la empresa ha optado por la versión 17 de Odoo debido a su naturaleza de código abierto y su costo cero. Este paso estratégico no solo refleja una decisión financiera prudente, sino también el reconocimiento del potencial de Odoo para impulsar el crecimiento y la eficacia operativa de la organización.